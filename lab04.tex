\chapter{The Competition Between Brownian Motion and Directed Forces}
\thispagestyle{fancy}
\fancyhead[RE,LO]{Experiment \thechapter}
\section{Random vs. Directed Motion.}
Much of the first month of Physics 2131 focused on directed motion.
In both the lecture and laboratory portions of the course, we saw that when the forces applied to objects did not vary substantially from moment to moment, the objects accelerated/established motion in one particular direction.
\par
Since then, we have begun to explore situations in which the pushes and pulls from the environment surrounding the object fluctuate rapidly in time — so rapidly that directed motion is no longer observed. 
We describe the motion of objects experiencing this volatile bombardment as ``random'' motion. 
In lab we have been exploring some of the qualitative and quantitative features of this random motion.
\par
While ``random'' motion and diffusion are important mechanisms, they propagate very slowly over long distances. 
Therefore, in many biological processes both random and directed motions are utilized. 
Over the next two weeks we will take a look at situations where both random and directed motion can be observed at the same time. 
We will try to determine the conditions under which one or the other type of motion is dominant.
\subsection*{Investigation}
Our overall goal here is to create conditions in which our beads undergo both random and directed motion, and to find a way to characterize that situation. 
We've provided you with metal ``steps'' that allow you to tilt the microscope to a couple of different angles. 
Be very careful when tilting the microscopes! 
Tilting the microscope changes the direction of the normal force exerted by the microscope slide on the beads. 
What happens to the motion of the beads when the microscope is tilted in this way? 
Because of the dangers of fluid flowing onto the microscope objective, we will be using petri dishes for this lab. 
\textsc{Be sure that you fill the chamber to the point where fluid covers the entire surface.
This way, we can minimize the amount of fluid flow that occurs, and look only at the motion of the beads relative to the fluid.}
\begin{enumerate}
\item Video \# 1: Record a video of 5 $\mu$m silica beads in water with the microscope tilted using the middle step of the block. A video of 2 to 5 seconds length with about 10 frames per second would be sufficient. Can you observe directed motion? If so, take a video for a long enough time interval that directed motion is clearly visible. Measuring the position of beads for at least 5 time points, create a plot that shows the dependence of $r^{2}$ on time interval for the beads. (If you are using automatic tracking (multitracker plugin), then use all produced data for all `real'\footnote{
A `real' track exists if the bead being followed: a) does not enter/leave the selected area over the course of the entire video; b) does not merge with or separate from another bead/track over the course of the entire video; c) does not blink in and out of existence; d) does not 'jump' (track number switching beads, causing a major discontinuity in x- and y-pixel locations); and e) exhibits some form of motion (i.e., is not stuck to the slide/chamber).} 
bead tracks—not just 5-10 time points.) Is the dependence of $r^{2}$ on time linear, as it was for the random motion when the microscope was not tilted?
\item Video \# 2: Next, take a sample containing 2 $\mu$m silica beads. Record a video of the bead's motion over the same time interval. Track the beads and plot the dependence of $r^{2}$ on time interval. Can you see the directed motion at all or are the beads merely displaying random motion?
\item Video \# 3: Now record a video of the 2 $\mu$m silica beads' motion over a larger time interval (a minute or two). For the sake of video size, reduce the frame rate to ~1 fps. Track the beads (using 10 to 15 time points) and plot the dependence of $r^{2}$ on time interval. What type of motion is visible now?
\end{enumerate}
\subsection*{Interpretation}
Some questions to consider answering in your lab report:
\begin{list}{-}{}
\item If the motion you are observing in some fluid is entirely random, how would you expect $r^{2}$ to vary as a function of the time interval over which you measure? 
If the motion you are observing in some fluid is entirely directed, how would you expect $r^{2}$ to vary as a function of the time interval over which you measure? 
\item Can a log-log plot help you distinguish between random and directed motion? 
Does the slope of the log-log plot change at all? 
Does the slope clearly indicate random or directed motion? 
\item Is it possible to have a situation where both types of motion occur? 
Which one would ‘win'? 
At what time scales? 
What are the conditions under which the motion of an object appears to be purely random? 
What are the conditions under which the motion of an object appears to be purely directed? 
\item Does the concentration of beads that you see on the slide when you look at it under the microscope change at all over the course of the experiment? 
Can you explain this effect using your results?
\end{list}