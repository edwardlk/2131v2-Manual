\chapter{The Competition Between Brownian Motion and Directed Forces}
\thispagestyle{fancy}
\fancyhead[RE,LO]{Experiment \thechapter}

Much of the first month of Physics 2131 focused on directed motion.
In both the lecture and laboratory portions of the course, we saw that when the forces applied to objects did not vary substantially from moment to moment, the objects accelerated/established motion in one particular direction.
\par
Since then, we have begun to explore situations in which the pushes and pulls from the environment surrounding the object fluctuate rapidly in time — so rapidly that directed motion is no longer observed. 
We describe the motion of objects experiencing this volatile bombardment as ``random'' motion. 
In lab we have been exploring some of the qualitative and quantitative features of this random motion.
\par
While ``random'' motion and diffusion are important mechanisms, they propagate very slowly over long distances. 
Therefore, in many biological processes both random and directed motions are utilized. 
Over the next two weeks we will take a look at situations where both random and directed motion can be observed at the same time. 
We will try to determine the conditions under which one or the other type of motion is dominant.

\paragraph{For this two week lab:} Our overall goal here is to create conditions in which our beads undergo both random and directed motion, and to find a way to characterize that situation. 
You will use beads of different sizes suspended in solution to explore the crossover from random to directed motion as an external force is applied.
%We've provided you with metal blocks that allow you to tilt the microscope to a couple of different angles.
Tilting the microscope changes the direction of the normal force exerted by the microscope slide on the beads, allowing gravity to have an effect on the beads diffusing around the surface of the petri dish.
Your goal is to observe the crossover from random motion (where thermal forces dominate) to directed motion (where gravity dominates) and analyze how it depends on the size of the bead:
\begin{enumerate}
\item Make sure that you understand how to safely operate the microscope, how to capture video with the microscope, and the pixel-to-distance ratio of the microscope camera.
\item Capture your first videos and do a preliminary analysis
\item Discuss as a class the method you plan to use to analyze your videos.
\item Collect the rest of your videos.
\item Analyze all of your videos to characterize the motion of the particles.
\end{enumerate}

\section{Random vs. Directed Motion.}
On the first day of this experiment, begin by determining what setup will best allow you to characterize directed motion. 
If the tilt is too low, there won't be enough force on the particles for you to see the directed motion; too much, and gravity will dominate over the random motion.
Start at a low tilt and move up until you can just begin to notice directed motion in your videos (either visually, or by doing a Multitracker analysis until you see tracks like those in figure \ref{fig:multitracker}).
\par
In order to tilt the microscope, you have been provided a few 3/8'' metal bars. 
\emph{Be careful when adjusting the tilt of the microscope!}
Have one member of the group tilt the microscope while another member places the steps under the feet.
\par 
In this experiment, you should investigate the behavior of both 2 $\mu m$ and 5 $\mu m$ silica beads in water.

\subsection*{Log-log Plots}
When we have a complicated function it is sometimes useful to approximate it by a simple power law.
One way to see how to do this is to use a \emph{log-log plot}.
That is, instead of just plotting the variables themselves, we plot the logarithm of the variables.
Let's see how this works.
Suppose we have a power law function $y = x^{N}$.
If we plot this, we get a curve like shown in figure \textbf{X}.
The more powers we have, the faster it rises (and the odd powers are negative for negative values of x.)
But if we take the logarithm of both sides of that equation, $y = x^{N}$,we get
\[ log(y)=N \cdot log(x) \] 
If we now take as new variables $Y = log(y)$ and $X = log(x)$, then our new equation is just $Y = N \cdot X$.
This is the graph of a straight line and the slope is proportional to the power.
If we plot this we get the figure at the right.
All the power laws are straight lines with increasing slope as the powers go up.
So if we have some complicated function that can be approximated by a power law, we can easily see that this is the case by plotting the logarithms of the variables. 
If we get a straight line a power law works. 
(We have only plotted positive values of x and y in the log-log plot since the log of a negative number is not a real number.)
Note that this works for negative powers too. 
Here's what the linear and log-log plots look like for these.
%\begin{enumerate}
%\item Video \# 1: A video of the 5 $\mu m$ silica beads, 5 seconds in length at 30 frames per second. Can you observe directed motion? If so, take a video for a long enough time interval that directed motion is clearly visible. Is the dependence of $r^{2}$ on time linear, as it was for the random motion when the microscope was not tilted?
%\item Video \# 2: Next, take a sample containing 2 $\mu$m silica beads. Record a video of the bead's motion over the same time interval. Track the beads and plot the dependence of $r^{2}$ on time interval. Can you see the directed motion at all or are the beads merely displaying random motion?
%\item Video \# 3: Now record a video of the 2 $\mu$m silica beads' motion over a larger time interval (2 minutes at 1 fps). Track the beads and plot the dependence of $r^{2}$ on time interval. What type of motion is visible now?
%\end{enumerate}
\subsection*{Interpretation}
Some questions to consider answering in your lab report:
\begin{enumerate}
\item Is it possible to have a situation where both types of motion occur? Which one would ‘win'? At what time scales? What are the conditions under which the motion of an object appears to be purely random? What are the conditions under which the motion of an object appears to be purely directed? 
\item If the motion you are observing in some fluid is entirely \textbf{random}, how would you expect $r^{2}$ to vary as a function of the time interval over which you measure?
\item If the motion you are observing in some fluid is entirely \textbf{directed}, how would you expect $r^{2}$ to vary as a function of the time interval over which you measure? 
\item Can a log-log plot help you distinguish between random and directed motion? Does the slope of the log-log plot change at all? Does the slope clearly indicate random or directed motion? 
%\item Does the concentration of beads that you see on the slide when you look at it under the microscope change at all over the course of the experiment? Can you explain this effect using your results?
\end{enumerate}