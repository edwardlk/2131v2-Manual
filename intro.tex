\clearpage
\setcounter{page}{1}

\chapter*{Introduction}
\thispagestyle{fancy}
\fancyhead[RE,LO]{Syllabus}
Most labs show you how to use equipment or follow instructions.
Sometimes their goal is to have you carry out a demonstration with your own hands instead of watching a lecturer do it – to convince you that something told to you is really the way things work.
None of this resembles real experimental science.
And since experiment is fundamental to science, we have created some labs that are trying to get you to get the flavor of a real scientific experiment. \par
These labs are designed to model the actual process of science, thus there are some unusual features that will differ from labs you may have attended in the past:
\begin{enumerate}
\item \textbf{The main goal of this class is to give you experience in designing an
experiment to answer a question.} Your goal will be to think about how you can
make a good measurement that answers the question. And you will need to think
about how the design of the experiment affects the certainty of your result.
\item \textbf{In some cases in this lab, you will do the experiment before it’s talked about
in lecture.} Sometimes in real life you don’t know what will happen and you have to
explore experimentally first. The idea here is to do an experimental to try to find out
what the behavior is without knowing the answer beforehand – just like many real
scientific experiments.
\item \textbf{It's not only your result that matters; it’s how good do you think it is –
quantitatively.} No experiment gives an exact result or the same exact result every
time it is repeated. You will have to decide not just ``the answer'' but what range of
values are that you think are possible given how you did the experiment.
\item \textbf{You will have to present your results to the rest of the class, have them
comment on your results, and comment on theirs.} Real science is a community
process. Every experiment is considered by others and often challenged. The process
of many people analyzing and thinking about everyone's work helps get to the real
answer and purge the ``wishful thinking'' we are all prone to.
\end{enumerate}

\section*{How to be successful in this course}
The key to being successful in this course is to focus on teamwork.
By working together to efficiently gather and analyze the data, your team will be free to spend more time discussing your results.
This is the true focus of this lab class: using your data to draw appropriate conclusions about the phenomena that you are investigating.
So how can you make the best use of your time to get a good grade in this class?
\begin{itemize}
	\setlength\itemsep{2pt}
	\item \textbf{Before you come to lab:} Read over the entire lab.
	Spend some time thinking about what is involved and how you might do the experiment.
	Read through any relevant sections in the textbook even if the lecture class hasn't reached them yet.
	\item \textbf{While you are in the lab:} Focus on your goals and tasks in the lab, don’t waste time.
	If you take a lot of time on irrelevancies, you may have trouble finishing.
	Remember to document what you are doing in a lab report created as you go.
	\item \textbf{When you leave the lab:} Be sure each team member has a copy of the data.
	You don’t want to arrive in the second (or third) week and find the only person with your data has dropped the course (or is sick, or is away at a sports event, or forgot it, or ...).
	At the end of a one-week lab or the last week of a multi-week lab, hand in your finished lab report before you leave.
	There is no out-of-class lab work required and no late-submissions are permitted.
\end{itemize}

\section*{Roles}
In order to facilitate the preparation of the lab report, you will be working in groups of two or three.  There are three roles that your group members will fill; while each member takes primary responsibility for one role and for the portion of the lab report related to that role, please keep in mind that the experiment is a group effort and you should all be aware of the dilemmas faced by your peers and the decisions that they make.  Also, except when writing the report, these lab experiments often involve ``all hands on deck'' -- with every group member contributing to the construction, execution, and analysis of an experimental protocol.  The division of labor will be as follows:
\begin{list}{}{\topsep=2pt \itemsep=1pt}
\item \textbf{The Journalist:} This person is primarily responsible for taking notes of everything that happens during the experiment and writing up the ``Introduction'' section of the lab report.
 
\item \textbf{The Data Interpreter:} This person primarily deals with tabulating and displaying the data, operating the computer, and writing up the ``Graphs \& Analysis'' section of the lab report. While all members of the group are expected to be involved in the collection and analysis of the results, this person is responsible for making them presentable.
 
\item \textbf{The Checker:} This person is responsible for making sure that the group is properly following the experiment plan, and makes sure that all requirements from the lab manual are being met. They are in charge of writing the ``Conclusion'' section. This person also acts as a ``manager'' of the lab tasks, stepping in where help is needed and coordinating the group's efforts to ensure the lab is completed efficiently and on-time.
\end{list}

\newpage

\section*{Lab Reports}
Each group will work together to submit one lab report for each experiment they complete.
The lab report must be submitted electronically to your lab TA at the end of each experiment, late submissions will not be accepted unless under extenuating circumstances.

\subsection*{Introduction \& Methods}
This section should begin with a description of the process you are investigating.
Describe what data you plan to gather and explain how it will be useful in your investigation.
Briefly describe the steps you took to gather that data and the methods you used to analyze your results. \\
\textbf{Requirements}
\begin{list}{-}{\topsep=0pt \itemsep=0pt}
	\item At least one paragraph in length
	\item Simple graphs may be used to illustrate the experimental setup
	\item Explain any equations that you will be using in the course of your data analysis
	\item Be detailed but avoid minutia! No one needs to read the full details of every single step you took. 
\end{list}

\subsection*{Graphs \& Analysis}
This section should include graphs of the data that you mentioned in the introduction.
The lab manual will guide you on the number of graphs that you should aim to include, but feel free to add more if you feel they are relevant.
After each graph, include a few sentences explaining it: point out important features, comment on the fit of the data against any theoretical curves, etc. \\
\textbf{Requirements}
\begin{list}{-}{\topsep=0pt \itemsep=0pt}
	\item All axes must be well labeled at a legible size and with appropriate units
	\item Use reasonable significant figures
	\item Calculate uncertainty whenever possible and use 
	\item Don't skimp out on the explanations after each graph, as this is probably the most critical part of any report. Don't rely on the reader to make the connection between what is in the graph and the phenomena you are investigating, tell them explicitly.
\end{list}

\subsection*{Conclusion}
In this section you want to demonstrate how all your results connect to elucidate the phenomena you set out to investigate.
Discuss the strongest and weakest areas of your investigation, and describe changes that you would make to your experimental plan if you were to repeat your work.
If possible, discuss how the things that you learned about in each experiment would be applicable to topics studied in other science classes.

\newpage
\section*{Grading}
The lab grade makes up part of your total course grade.
This grade will be based on your team's lab reports and your individual participation in the lab and the class discussions.
\emph{Your grade will not depend on whether or not your numerical results agree with some accepted standard but on how well you conceived and carried out the experiment.} \\

\begin{tabular}{|p{13cm}|c|}
\hline
\textbf{Team Lab Report} & \textbf{Weight} \\
\hline
\emph{Design and thoughtfulness:} Did your team do a careful and thoughtful job in creating your experiment, and was this thought reflected in the journal? & 20\% \\
\hline
\emph{Clarity and completeness:} Did your team explain your experiment so that someone could reproduce it? & 20\% \\
\hline
\emph{Persuasiveness:} What conclusions did your team draw from your data and were you able to back up these conclusions with this data in a convincing way? & 20\% \\
\hline
\emph{Evaluation:} After observing the experiments of other groups, were you able to critique your own lab, propose constructive changes, or explain why your experiment was better than those of your classmates?  (The question you are answering in your evaluation is, ``If I got to re-do this experiment next week, how would I do it differently?'') & 20\% \\
\hline
\textbf{Individual Participation} & \\
\hline
\emph{Contribution to team presentation:} Did you participate constructively in your own group's work (protocol development and data collection)?  Did you actively participate in both the preparation of the report/presentation and its delivery? & 10\% \\
\hline
\emph{Contribution of other teams' presentations:} Did you ask useful questions or make comments that were valuable to the other teams' reports of their evaluations?  Did you participate in both class and small group discussion? & 10\% \\
\hline
\end{tabular}

\newpage

\section*{Schedule}
Below is a preliminary schedule for the Winter 2017 semester.
This schedule is subject to change due to unforeseen circumstances, any changes will be announced via blackboard. \\

\begin{tabular}{ |c|c|l| } 
 \hline
 \textbf{Week} & \textbf{Week Begins} & \textbf{Content} \\ 
 \hline
    & Jan. 9 & \textbf{No Class} \\ 
 \hline 
%  1 & Jan. 9 & Introduction \& Equipment walkthrough \\ 
% \hline 
    & Jan. 16 & \textbf{No Class} \\ 
 \hline
 1 & Jan. 23 & Experiment 1.1 \\ 
 \hline
 2 & Jan. 30 & Experiment 1.2 \\ 
 \hline
 3 & Feb. 6 & Experiment 2.1 \\ 
 \hline
 4 & Feb. 13 & Experiment 2.2 \\ 
 \hline
 5 & Feb. 20 & Experiment 3.1 \\ 
 \hline
 6 & Feb. 27 & Experiment 3.2 \\ 
 \hline
 7 & March 6 & Experiment 3.3 \\ 
 \hline
   & March 13 & \textbf{Spring Break (No Class)} \\ 
 \hline
 8 & March 20 & Experiment 4 \\ 
 \hline
 9 & March 27 & Experiment 4 (cont.) \\ 
 \hline
 10 & April 3 & Experiment 5 \\ 
 \hline
 11 & April 10 & Experiment 5 (cont.) \\ 
 \hline
 12 & April 17 & Exam \\ 
 \hline
% 15 & April 24 & cell9 \\ % Monday (4/24) is last day of semester
% \hline
\end{tabular}

\section*{Late/Absentee Policy}
If you anticipate missing a lab session, try to arrange ahead of time to attend another lab section for that session or for the entire lab unit.
If it is not possible to attend a different lab session, contact your TA as soon as you are aware of your impending absence.
Only those with a \textbf{VALID WRITTEN EXCUSE} for missing a lab will be allowed to do a makeup activity at the end of the semester (that will take at least two hours and may involve doing another lab or evaluating data).
If you do not have a valid written excuse, you will get a zero for the week that you missed.
You may make up a maximum of one excused absence.
\textbf{If you miss more than one week (have more than one `zero', i.e., if you miss more than one lab session), you may receive an incomplete or a failing grade for the entire class.}

\newpage

\section*{Students with Disabilities}
If you have a documented disability that requires accommodations, you will need to register with Student Disability Services for coordination of your academic accommodations. 
The Student Disability Services (SDS) office is located at 1600 David Adamany Undergraduate Library in the Student Academic Success Services department. 
SDS telephone number is 313-577-1851 or 313-577-3365 (TTD only). 
Once you have your accommodations in place, I will be glad to meet with you privately during my office hours or at another agreed upon time to discuss your needs. 
Student Disability Services' mission is to assist the university in creating an accessible community where students with disabilities have an equal opportunity to fully participate in their educational experience at Wayne State University.