\clearpage
\setcounter{page}{1}

\chapter*{Introduction}
\thispagestyle{fancy}
\fancyhead[RE,LO]{Syllabus}
Most labs show you how to use equipment or follow instructions.
Sometimes their goal is to have you carry out a demonstration with your own hands instead of watching a lecturer do it – to convince you that something told to you is really the way things work.
None of this resembles real experimental science.
And since experiment is fundamental to science, we have created some labs that are trying to get you to get the flavor of a real scientific experiment. \par
These labs are designed to model the actual process of science, thus there are some unusual features that will differ from labs you may have attended in the past:
\begin{enumerate}
\item \textbf{The main goal of these labs are to give you experience in designing an
experiment to answer a question.} Your goal will be to think about how you can
make a good measurement that answers the question. And you will need to think
about how the design of the experiment affects the certainty of your result.
\item \textbf{In some cases in this lab, you will do the experiment before it’s talked about
in lecture.} Sometimes in real life you don’t know what will happen and you have to
explore experimentally first. The idea here is to do an experimental to try to find out
what the behavior is without knowing the answer beforehand – just like many real
scientific experiments.
\item \textbf{It's not only your result that matters; it’s how good do you think it is –
quantitatively.} No experiment gives an exact result or the same exact result every
time it is repeated. You will have to decide not just ``the answer'' but what range of
values are that you think are possible given how you did the experiment.
\item \textbf{You will have to present your results to the rest of the class, have them
comment on your results, and comment on theirs.} Real science is a community
process. Every experiment is considered by others and often challenged. The process
of many people analyzing and thinking about everyone's work helps get to the real
answer and purge the ``wishful thinking'' we are all prone to.
\end{enumerate}

\section*{Late/Absentee Policy}
If you anticipate missing a lab session, try to arrange ahead of time to attend another lab section for that session or for the entire lab unit.
If it is not possible to attend a different lab session, contact your TA as soon as you are aware of your impending absence.
Only those with a VALID WRITTEN EXCUSE for missing a lab will be allowed to do a makeup activity at the end of the semester (that will take at least two hours and may involve doing another lab or evaluating data).
If you do not have a valid written excuse, you will get a zero for the week that you missed.
You may make up a maximum of one excused absence.
\textbf{If you miss more than one week (have more than one 'zero', i.e., if you miss more than one lab session), you may receive an incomplete or a failing grade for the entire class.}

\section*{Lab Reports}
Each student will be required to submit their own lab report for each experiment they complete.  

\subsection*{Introduction}
Prior to the first day of each experiment, each student is expected to read through the lab manual and write an introduction to the experiment they are about to perform.
\paragraph*{Requirements}
\begin{list}{-}{}
\item At least one paragraph in length
\end{list}

\subsection*{Graphs \& Analysis}
At the end of each experiment, each student is expected to graph and analyze their data.
\paragraph*{Requirements}
\begin{list}{-}{}
\item All axes must be well labeled at a legible size and with appropriate units 
\end{list}  

\section*{Grading}

\begin{tabular}{ |l|c|c| } 
 \hline
 \multicolumn{2}{|c|}{\textbf{Assignment}} & \textbf{Weight} \\ 
 \hline
 \multirow{2}{*}{Lab Reports} & Introduction & 20\% \\
 \cline{2-3}
                              & Graphs \& analysis & 50\% \\ 
 \hline 
 \multicolumn{2}{|l|}{Participation} & 10\% \\ 
 \hline
 \multicolumn{2}{|l|}{Data Presentation} & 20\% \\ 
 \hline
\end{tabular}

\newpage 

\section*{Schedule}

\begin{tabular}{ |c|c|l| } 
 \hline
 \textbf{Week} & \textbf{Week Begins} & \textbf{Content} \\ 
 \hline
% 1 & Jan. 9 & Introduction \& Equipment walkthrough \\ 
% \hline 
 1 & Jan. 23 & Experiemnt 1.1 (meets in computer lab room 371) \\ 
 \hline
 2 & Jan. 30 & Experiment 1.2 (meets in computer lab room 371)\\ 
 \hline
 3 & Feb. 6 & Experiment 2.1 \\ 
 \hline
 4 & Feb. 13 & Experiment 2.2 \\ 
 \hline
 5 & Feb. 20 & Experiment 3.1 \\ 
 \hline
 6 & Feb. 27 & Experiment 3.2 \\ 
 \hline
 7 & March 6 & Experiment 3.3 \\ 
 \hline
% 9 & March 6 & Lab Make-up day \\ 
% \hline
   & March 13 & \textbf{Spring Break (No Class)} \\ 
 \hline
 8 & March 20 & Experiment 4 \\ 
 \hline
 9 & March 27 & Experiment 4 (cont.) \\ 
 \hline
 10 & April 3 & Experiment 5 \\ 
 \hline
 11 & April 10 & Experiment 5 (cont.) \\ 
 \hline
 12 & April 17 & Lab Make-up day \\ 
 \hline
% 15 & April 24 & cell9 \\ % Monday (4/24) is last day of semester
% \hline
\end{tabular}

\section*{Students with Disabilities}
If you have a documented disability that requires accommodations, you will need to register with Student Disability Services for coordination of your academic accommodations. 
The Student Disability Services (SDS) office is located at 1600 David Adamany Undergraduate Library in the Student Academic Success Services department. 
SDS telephone number is 313-577-1851 or 313-577-3365 (TTD only). 
Once you have your accommodations in place, I will be glad to meet with you privately during my office hours or at another agreed upon time to discuss your needs. 
Student Disability Services' mission is to assist the university in creating an accessible community where students with disabilities have an equal opportunity to fully participate in their educational experience at Wayne State University.