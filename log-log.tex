\chapter{Log-Log Plots}
\thispagestyle{fancy}
\fancyhead[RE,LO]{Technical Document \thechapter}

Log-Log plots have many uses in science.
They are especially helpful for differentiating between different forms of \emph{functional dependency}.
What is functional dependency? 
Functional dependency tells us how one quantity varies when another is adjusted. 
Example: for purely random motion, the diffusion of an object in 2-D space has a functional dependence like $\langle r^{2} \rangle = 4 D \Delta t$. 
As you saw in lab, the viscosity and temperature of the fluid medium surrounding the object and the size of the object can affect the value of the diffusion constant, D, and thus the linear slope of the $\langle r^{2} \rangle$ vs.\ $\Delta t$ plot (where the slope is related to D). 
A $log(\langle r^{2} \rangle)$ vs.\ $log(\Delta t)$ plot of this functional dependency would be a line with a slope of 1 — regardless of the value of the diffusion constant, D! 
If we are not learning the diffusion constant D from this log-log plot, what does the slope of 1 tell us?
\par 
It turns out that the slope of $log(\langle r^{2} \rangle)$ vs.\ $log(\Delta t)$ tells us about what type of motion we are observing! 
Most motion will not be linear in an $\langle r^{2} \rangle$ vs.\ $\Delta t$ plot. 
There are other functional dependencies that can exist governing the relationship between $\langle r^{2} \rangle$ and $\Delta t$. 
For directed motion at constant velocity, $\langle r^{2} \rangle$ changes as $(\Delta t)^{2}$ and so the $\langle r^{2} \rangle$ vs.\ $\Delta t$ plot would be quadratic. 
But a $log(\langle r^{2} \rangle)$ vs.\ $log(\Delta t)$ plot of this motion is still linear, with a slope of 2. 
For directed motion under uniform acceleration from rest, the distance traveled is $r = \frac{1}{2} a \Delta t^{2}$, so $\langle r^{2} \rangle$ changes linearly with $(\Delta t)^{4}$ — thus a $log(\langle r^{2} \rangle)$ vs.\ $log(\Delta t)$ plot of this motion has a slope of 4!
\par 
Other motion in cells is confined, e.g. molecules that are caged by a surrounding scaffolding of actin, or molecules on the membrane that are confined to a ``lipid raft'' or patch of membrane that has some functional activity. 
For such caged motion, $\langle r^{2} \rangle$ does not quite increase linearly with $\Delta t$ — the distance moved gets smaller than we would expect for random motion as we get to larger and larger distances — thus a $log(\langle r^{2} \rangle)$ vs.\ $log(\Delta t)$ plot of this motion has a slope of less than 1.
 \par
For real biological systems, the motion is often a combination of random and directed motion — and thus a $log(\langle r^{2} \rangle)$ vs.\ $log(\Delta t)$ plot of this motion has a slope between 1 and 2. 
The dominant functional dependency (the one that determines the type of motion we observe) can also depend on the time scale at which we observe the motion.
\par
This is all very interesting, but how is it helpful to us?
In biology, almost all motion can look quite random, but that apparently random motion can hide other processes that may look similar to random motion — in particular, caged motion:
\begin{itemize}
\item In ecology, tracking data from tagged animals can be analyzed to find the roaming grounds of an animal. However, to distinguish random motion from the confined area to which an animal intentionally returns, we cannot simply look at the tracks by eye. A log-log plot can help us find the distances at which the motion gets confined; helping us determine the size of roaming grounds and how often roaming grounds are changed. In addition, on shorter timescales the motion will look directed since animals are able to move straight (at least over short distances)!
\item On much smaller scales, within cells, the tracking data from individual molecules on a cell membrane have helped develop the theory of ``lipid rafts''. Such rafts are patches of membrane that float on the overall cell membrane. Within the raft sit a number of functional molecules that appear to operate together, taking advantage of their closeness to each other within a raft to enhance signals. The significance of lipid rafts in biology is still under investigation and log-log plots are a key tool to distinguish randomness from caged motion!
\end{itemize}
%
Below is a chart to help summarize the broad categories of functional dependency and their effects on the slope of the $log(\langle r^{2} \rangle)$ vs.\ $log(\Delta t)$ plots.

\begin{table}[h!]
	\centering
	\begin{tabular}{|l|c|}
	\hline 
	Type of motion & Slope of the $log(\langle r^{2} \rangle)$ vs.\ $log(\Delta t)$ plot \\ 
	\hline 
	Confined & $s < 1$ \\ 
	\hline 
	Random & $s = 1$ \\ 
	\hline 
	Biological & $1 < s < 2$ \\ 
	\hline 
	Directed & $s = 2$ \\ 
	\hline 
	Accelerated & $s > 2$ \\ 
	\hline 
	\end{tabular} 
	\caption{The effect of movement type on a log-log plot.}
	\label{tab:logPlt}
\end{table}
These types of plots can also help us model a new phenomenon. 
Imagine a situation in which you wish to develop a model of how a biological or chemical process spreads in space with increasing time. 
Knowing the functional dependency between $\langle r^{2} \rangle$ and $\Delta t$ can help you build a model or help you choose between competing models. 
Once we understand the model better, we can make predictions about what will happen when we make changes to the system. 
In order to help us with our modeling, we take data on the motion and make a plot of $log(\langle r^{2} \rangle)$ vs.\ $log(\Delta t)$. 
The slope of this plot helps us make decisions about what models to propose or what models to eliminate. 
It also tells us which functional dependency dominates at which time scales — it tells us when a type of motion is the most important and when we can ignore the effects of other types of motion.
\par
(Note that log-log plots help with functional dependencies more generally. 
In chemistry, you may need to plot the log of a molecule's solubility vs.\ the pH.
Since the pH is the log of the number of free hydrogen ions, this is again a log-log plot in disguise! 
Then the slope can tell you how many charges an atom or molecule has in solution.)