\chapter{Inferring Force Characteristics from Motion Analysis}
\thispagestyle{fancy}
\fancyhead[RE,LO]{Experiment \thechapter}
%
\section{Introduction to video capture \& analysis of directed motion and resistive forces.}
This is the first week of a two-week lab sequence designed to introduce you to video capture and analysis of directed motion and resistive forces.
In this first week, you will collect video data using ImageJ for two separate investigations.
Next week you will analyze your data and try to determine how the resistive forces scale with respect to the varied quantities.
In the first investigation, you will be asked to analyze the directed motion of different spheres falling through fluid (one concentration of glycerol).
(You are investigating how resistive forces and terminal velocity scale with the mass/size of the falling object.)
In the second investigation, you will analyze the directed motion of one sphere falling through different fluids (different concentrations of glycerol).
(You are investigating how resistive forces and terminal velocity scale with the viscosity of the fluid.)
Make sure that all students are using the same size sphere in the second investigation.
%The lab handout will give explicit instructions on video capture, but no guidance on the performance of the experiments or attainment of the physics skill goals. 
\par
Why do we care about resistive forces and directed motion?
The resistive effect of air (on macroscopic organisms) and fluids (on both macroscopic and microscopic organisms) cannot always be ignored.
For example, the fluid resistance on a bacterium (or on a cell) requires that the bacterium exert a force to overcome this resistance and change its motion.
These resistive forces can be affected by the size and mass of an object and by the characteristics of the medium creating the resistive force.
You will learn more about this in the upcoming lectures, readings, and recitation.
Pay attention, as you will need these theoretical ideas to do your data analysis next week. 
\par 
First, you will practice and master the skills necessary to capture and edit your own videos for analysis in ImageJ.
It is advised that you carry out one investigation from beginning to end (video capture, video editing, \& imageJ analysis) before you capture the rest of your videos.
After today, you will ALL be expected to be experts at these skills so take turns and help each other learn.
Take notes for the future if you are worried that you will forget.

\subsection*{*Important Note on Video Capturing*}
Recording high quality videos using the methods detailed below can result in very large video files. Capturing one 30-second video can create a file as large as \emph{1.5 GB}; if we are not careful about the amount of video that is being captured, we can quickly overfill the hard drive and crash the computer. Therefore, it is important that you follow the steps as they are listed below. Setting the stop conditions in step 5 ensures that in case you forget to end the video capture, the program would stop capturing on its own. In step 3, you are asked to use a common file to store the videos that you capture. Occasionally, the TAs will need to delete all the videos in this folder to make room on the hard drive. Therefore, it is important to edit the videos that you capture to save only the frames that contain data you plan on analyzing, and then save these smaller video files to your group folder. Please check with your TA if you are unsure about how to do this.

\subsection{Capturing a .avi Video}
\begin{enumerate}
\item The program that we will be using to capture videos for analysis is called `VirtualDub', a shortcut to it can be found on the desktop. After double-clicking the shortcut, a video editing-editing window opens. Navigate to `File $>$ Capture AVI...' to get the program into capture mode.
\item Click `Device $>$ Microsoft LifeCam Studio(TM) (DirectShow)' to show the output from the camera. If you are not getting an image, ask your TA for help (sometimes shutting off the program, unplugging then re-plugging in the camera, and then restarting the program works).
\item Go to `File $>$ Set capture file...' , choose the folder `VirtualDub captures', and choose a file names for the videos you capture. Good file names end with numbers the program can increment, so use something like `GroupName-lab2-001' as your file name.
\item Uncheck `Audio $>$ Enable audio capture'.
\item Navigate to `Capture' and make sure the following settings are correct:
\begin{itemize}
	\item `Stop conditions...': check the box next to `capture time exceeds 30 seconds'. Click Accept.
	\item `Disk I/O...': set `Chunk Size' to 2 MB and `Chunks in buffer' to 20. Click Accept.
	\item Check `Autoincrement filename after capture'.
\end{itemize}
\item Now you are ready to begin capturing your videos. Align all the parts of your experiment and then either click `Capture $>$ Capture video' or press `F6'. To stop a capture, press `Esc'.
\end{enumerate}

\section{Error propagation and analysis of directed motion \& resistive forces}
%This is the second week of a two-week lab sequence designed to introduce you to error propagation and analysis of directed motion and resistive forces.
%In the first week, you collected video data using ImageJ for two separate investigations.
%In the first investigation, you gathered data investigating how resistive forces and terminal velocity scale with the mass of the falling object.
%In the second investigation, you gathered data investigating how resistive forces and terminal velocity scale with the viscosity of the fluid.
This week you will collect and analyze the rest of your data and try to determine how the terminal velocities scale with respect to the varied quantities.
%The lab handout will give explicit instructions on error propagation, but no guidance in the performance of the physics skill goals. \par
Below, you will find a table detailing the viscosities of the percent solutions that you worked with throughout this experiment.
Other possibly helpful information is available from your TA—if there is information (physical data) that you think you will need, don't hesitate to ask—but you have to know what you are asking for! \par
At the end of the lab today, your group will submit one lab report.
This will be reviewed by the TA according to the Scientific Community Lab rubric.
Good attention to detail now will save you time later!
Remember, your TA is here to help you with equipment, error propagation, and ImageJ, but the physics is up to you and your group!
\bigskip 
\begin{table}[h]
\centering
\resizebox{\textwidth}{!}{%
\begin{tabular}{@{}lccccccc@{}}
\toprule
Percent Glycerol          & 0\%    & 30\%   & 40\%   & 50\%   & 60\%   & 70\%   & 80\%   \\ \midrule
Dynamic Viscosity ($Ns/m^{2}$) & 0.0009 & 0.0026 & 0.0041 & 0.0070 & 0.0132 & 0.0278 & 0.4346 \\ \bottomrule
\end{tabular}%
}
\caption{Viscosity of glycerol concentrations by \% volume}
\label{tab:viscosity}
\end{table}

\paragraph{Summary of Results:\\} 
Before you begin writing up your lab report, you should have collected the following data:
\begin{itemize}
\item Videos of three or more different spheres falling through water.
\item Videos of one sphere falling through three or more different glycerol concentrations.
\end{itemize}

\paragraph{Things you might consider including in your report:}
(These are not necessarily required. You and your group should consider what INFORMATION is necessary to SUPPORT the CLAIMS that you are making!)
\begin{itemize}
\item Plots of y vs. t OR v vs. t for each video (Do you need both plots for each video?).
\item Data table for terminal velocity, terminal velocity squared, and m for investigation one.
\item Data table for terminal velocity and viscosity for investigation two.
\item Plots of terminal velocity vs. mass and terminal velocity squared vs. mass with error bars.
(Do you need both graphs?)
\item Plot of terminal velocity vs. viscosity with error bars.
\item Consider discussing the following questions:
	\begin{itemize}
	\item What is the terminal velocity for each video? How certain is this value?
	\item How do the different terminal velocities for each investigation fit together to describe the resistive force?
	\item For the 2nd investigation, how does terminal velocity scale with viscosity?
	\end{itemize}
\end{itemize}

%\paragraph{Summary of Results:}
%\begin{itemize}
%\item A graph showing the position of each sphere as a function of time.
%\item A graph showing the position of the sphere in each fluid as a function of time.
%\item From the graphs, determine the average acceleration of each sphere.
%\end{itemize}