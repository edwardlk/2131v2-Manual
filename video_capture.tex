\chapter{Introduction to Video Capture}
\thispagestyle{fancy}
\fancyhead[RE,LO]{Technical Document \thechapter}
\section*{Finding the software on the lab computer}
The software should be clearly available on the desktop. 
If it is not, go to the start menu and type `virtual' into the Search Programs and Files box. 
The software you want is called `VirtualDub' and has an icon that looks like a gray gear-and-screw. 
Open this.
\section*{Capturing video with the webcam}
\begin{itemize}
\item The fist window that opens is for editing videos. Select `File $>$ Capture AVI' to open the video capture window.
\item If both a webcam and a microscope are connected to your computer, you may need to select the device. 
Select `Device,' and `Microsoft LifeCam Studio (TM) (DirectShow)' for the webcam or the `Toupcam (DirestShow)' option for the microscope.
\item If you do not see the output from the camera showing in the window, make sure that `Video $>$ Preview' is checked.
\item There are a number of settings that you can investigate to change aspects of the video you wish to capture.
Your lab station will have a guide with more information about the correct settings to use, what settings you may need to adjust, and what settings should not be changed.
\item Be aware that ImageJ can process only about 400 frames (due to the limited working memory of the computer, more frames possible with the virtual stack option), so you will need to adjust the play rate and total time of your video to take the minimum length to capture your event and not much else. This can be done very easily in the VirtualDub editing window, see the guide in lab for more information.
\item Check the save location, file name, and auto-increment settings; set these to your group;s data folder so that you don't lose any data and avoid writing over anyone else's data.
\item When you are ready to capture video of motion/your subject, select `Capture video' or press F6. Once your experiment is complete, click `Stop Capture' or press `Esc' to stop recording. 
\item Keep in mind, you may need to collect several trial videos before you manage to get good video for analysis in ImageJ.
\end{itemize}
\section*{Capturing video with the microscope CCD camera}
Unfortunately, the user interface for the microscope camera is not quite as friendly as the webcam. 
In fact the frame rate, which is very important for the measurements you will be doing, is difficult to set exactly. 
\par 
The capture process for the microscope camera is very similar to the webcam. 
The main difference are in how to set the video settings, such as resolution, frame rate, and brightness. 
To adjust the output size (resolution, select ‘Video,’ ‘Capture Pin’). 
There should be three options available, but keep in mind that higher resolutions lead to drastically larger files. 
To adjust brightness and frame rate, select ‘Video,’ ‘Capture Filter’. 
Because of the nature of the camera, the best way to control the frame rate is to change the exposure time. 
The exposure time value is around the spacing  between frames, so 0.5 s exposure time will translate to around 2fps. 
Changing the exposure time will also change the brightness of the image, as more light will be collected for each frame. 
After you set the exposure time and take a video, it is important to record the resulting frame rate. 
You can find the frame rate in the information panel on the right of the video capture screen, under the video tab and next to the label “average rate.” 
\par 
The remainder of the video capture process for the microscope is identical to the webcam process. 
Be sure to change the name for each video to something that helps you identify the video, and don’t forget to record the frame rate!
\section*{Video construction \& planning}
Creating a good video for analysis in ImageJ is tougher than you might think. 
You will need to consider elements of video construction and planning. 
You need to be a ‘Good Cinematographer.’ 
Here are some questions to consider:
\begin{itemize}
\item What is the best angle? How should the camera be aligned to view the motion of the object
in which you are interested?
\item What is the best time between frames? How many frames-per-second should you be collecting, given the time for the phenomena to occur and the memory limitations of ImageJ (about 400 frames)?
\item Is there a known length visible in the video?
\item Are all objects of interest clearly visible in the video?
\item Is the entire portion of motion in which we are interested visible in the video?
\item Is the camera (perspective) stationary?
\end{itemize}
\section*{Determining distance-to-pixel ratio in ImageJ}
Once you have imported the .AVI file into ImageJ, you will need to determine the distance-to-pixel ratio for your video (this should be done separately for each video analyzed). 
\par 
You will need to use the 'line' tool (*Straight* tool, 5th from left end of icons in toolbar) in ImageJ. 
Click on the 'line tool' icon of the ImageJ menu toolbar (it looks like a sloped line or a slanted fraction bar (/), and the bottom right corner has a downward-pointing black triangle). 
Using this tool, click on one end of a ‘known length’ object hold down the mouse button, and drag the line across the ‘known length’ object to the other side. 
Without releasing the mouse button, look to see the length (in pixels) of your line segment--this information should be at the top of the video window or at the bottom edge of the ImageJ menu toolbar. 
It will say "length=.....". 
This length, in pixels, is equal to the ‘known length’ of the object. 
Now you have a distance-to-pixel ratio to help you turn pixel locations into physical positions. 
Once the mouse button is released, this pixel length will no longer be displayed by ImageJ. 
It can be recovered by selecting ‘Analyze,’ ‘Measure’ and looking at the measurement on the data table that appears.