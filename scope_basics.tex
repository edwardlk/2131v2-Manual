\chapter{Microscope Basics}
\thispagestyle{fancy}
\fancyhead[RE,LO]{Technical Document \thechapter}

\section*{Eyepieces:}
\begin{itemize}
	\setlength\itemsep{1pt}
	\item Adjustable to fit both eyes
	\item Sometimes easier if one eye is closed
	\item If eyelashes obstruct view, move closer to the eyepieces
\end{itemize}

\section*{Light Source, Condenser, Phase Contrast Panel:}
\begin{itemize}
	\setlength\itemsep{1pt}
	\item Condenser concentrates light from illumination source
	\item Phase rings in front of the light source allow the microscope to translate phase shifts in light that goes through a transparent sample into brightness changes in the observed images. This allows for very useful imaging of transparent samples that would be difficult with standard bright field imaging.
	\item The light intensity can be controlled by adjusting the orange wheel on the bottom left of the machine.
\end{itemize}

\section*{Diaphragm:}
\begin{itemize}
	\setlength\itemsep{1pt}
	\item Lever controls an iris, allowing the user to control the amount of light hitting the sample
	\item Very often, higher detail can be observed by allowing less light through the iris
\end{itemize}

\section*{Stage:}
\begin{itemize}
	\setlength\itemsep{1pt}
	\item Place the sample slide on the microscope stage carefully
	\item The stage can be moved relative to the objectives using the dials below the stage to the right
	\item Align the sample directly over the objective, so the light is shining directly on the sample
\end{itemize}

\section*{Objectives:}
\begin{itemize}
	\setlength\itemsep{1pt}
	\item Objectives collect the light from the samples and focus it to form an image in the eyepiece or CCD camera.
	\item Rotating turret holds four different objective lenses: 4X, 10X, 20X, and 40X
	\item Always start with a low magnification objective: find, center and focus on the sample before moving on to a higher magnification.
	\item Always lower objectives using the coarse adjustment knob before rotating the turret
\end{itemize}

\section*{Coarse and Fine Focusing Knobs:}
\begin{itemize}
	\setlength\itemsep{1pt}
	\item Bring sample into view using the coarse adjustment knob (inner knob)
	\item Once sample is in view, use the fine adjustment knob (outer knob) to achieve the sharpest possible image.
	\item Be careful when using the coarse adjustment knob, hitting the slide with the objective can scratch the objective or crack the slide.
\end{itemize}

\section*{CCD Camera:}
\begin{itemize}
	\setlength\itemsep{1pt}	
	\item Camera can be rotated using the adjustable pin below the lens
\end{itemize}